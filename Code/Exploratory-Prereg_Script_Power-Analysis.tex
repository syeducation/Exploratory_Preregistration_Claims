% Options for packages loaded elsewhere
\PassOptionsToPackage{unicode}{hyperref}
\PassOptionsToPackage{hyphens}{url}
\documentclass[
]{article}
\usepackage{xcolor}
\usepackage[margin=1in]{geometry}
\usepackage{amsmath,amssymb}
\setcounter{secnumdepth}{-\maxdimen} % remove section numbering
\usepackage{iftex}
\ifPDFTeX
  \usepackage[T1]{fontenc}
  \usepackage[utf8]{inputenc}
  \usepackage{textcomp} % provide euro and other symbols
\else % if luatex or xetex
  \usepackage{unicode-math} % this also loads fontspec
  \defaultfontfeatures{Scale=MatchLowercase}
  \defaultfontfeatures[\rmfamily]{Ligatures=TeX,Scale=1}
\fi
\usepackage{lmodern}
\ifPDFTeX\else
  % xetex/luatex font selection
\fi
% Use upquote if available, for straight quotes in verbatim environments
\IfFileExists{upquote.sty}{\usepackage{upquote}}{}
\IfFileExists{microtype.sty}{% use microtype if available
  \usepackage[]{microtype}
  \UseMicrotypeSet[protrusion]{basicmath} % disable protrusion for tt fonts
}{}
\makeatletter
\@ifundefined{KOMAClassName}{% if non-KOMA class
  \IfFileExists{parskip.sty}{%
    \usepackage{parskip}
  }{% else
    \setlength{\parindent}{0pt}
    \setlength{\parskip}{6pt plus 2pt minus 1pt}}
}{% if KOMA class
  \KOMAoptions{parskip=half}}
\makeatother
\usepackage{graphicx}
\makeatletter
\newsavebox\pandoc@box
\newcommand*\pandocbounded[1]{% scales image to fit in text height/width
  \sbox\pandoc@box{#1}%
  \Gscale@div\@tempa{\textheight}{\dimexpr\ht\pandoc@box+\dp\pandoc@box\relax}%
  \Gscale@div\@tempb{\linewidth}{\wd\pandoc@box}%
  \ifdim\@tempb\p@<\@tempa\p@\let\@tempa\@tempb\fi% select the smaller of both
  \ifdim\@tempa\p@<\p@\scalebox{\@tempa}{\usebox\pandoc@box}%
  \else\usebox{\pandoc@box}%
  \fi%
}
% Set default figure placement to htbp
\def\fps@figure{htbp}
\makeatother
\setlength{\emergencystretch}{3em} % prevent overfull lines
\providecommand{\tightlist}{%
  \setlength{\itemsep}{0pt}\setlength{\parskip}{0pt}}
\usepackage{bookmark}
\IfFileExists{xurl.sty}{\usepackage{xurl}}{} % add URL line breaks if available
\urlstyle{same}
\hypersetup{
  pdftitle={Evaluating Claims that Registration Restricts Exploratory Research},
  pdfauthor={Moin Syed},
  hidelinks,
  pdfcreator={LaTeX via pandoc}}

\title{Evaluating Claims that Registration Restricts Exploratory
Research}
\author{Moin Syed}
\date{2025-11-29}

\begin{document}
\maketitle

\section{Power Analysis}\label{power-analysis}

We can use O'Mahony (2023) as a starting point for the power analysis.
No effect size is reported, but we can calculate it using two methods.
The first method is based on the reported test statistic:

\begin{verbatim}
## 
##      Chi squared power calculation 
## 
##               w = 0.1781715
##               N = 303.5016
##              df = 2
##       sig.level = 0.05
##           power = 0.8
## 
## NOTE: N is the number of observations
\end{verbatim}

This yields a value of w = 0.178, which would require a sample of 304.
An alternative approach is to use the reported frequencies to
reconstruct the 2x2 contingency table, and use that as the input:

\begin{verbatim}
##      No  Yes
## SR 0.29 0.38
## R  0.08 0.25
\end{verbatim}

\begin{verbatim}
## 
##      Chi squared power calculation 
## 
##               w = 0.1854456
##               N = 280.159
##              df = 2
##       sig.level = 0.05
##           power = 0.8
## 
## NOTE: N is the number of observations
\end{verbatim}

This yields a value of w = 0.185, which is similar but just a little bit
larger. In this case, a sample size of 280 would be required.

O'Mahony's (2023) study included twice as many traditional reports as
Registered Reports. Next we examine what happens if we make the two
equal, but preserve the same relative frequencies:

\begin{verbatim}
##      No  Yes
## SR 0.21 0.29
## R  0.12 0.38
\end{verbatim}

\begin{verbatim}
## 
##      Chi squared power calculation 
## 
##               w = 0.1914027
##               N = 262.9913
##              df = 2
##       sig.level = 0.05
##           power = 0.8
## 
## NOTE: N is the number of observations
\end{verbatim}

Doing this results in an effect size of w = 0.191 and a necessary sample
size of 263.

All of the preceding are 2x2 contingency tables, and thus would assume
that we would collapse the two categories of registration (Registered
Reports and preregistered studies) and the two categories of traditional
articles (companion articles and 2010 articles). If, instead, we
separated the four articles types for analysis, there are a variety of
relevant scenarios worth probing with respect to their impact on the
required sample size.

In Scenario 1, we assume equal frequencies within the two categoroies
(registered and not), at the same rate as found in O'Mahony (2023):

\begin{verbatim}
##       SR   SO   RR   PR
## No  0.11 0.11 0.06 0.06
## Yes 0.14 0.14 0.19 0.19
\end{verbatim}

\begin{verbatim}
## 
##      Chi squared power calculation 
## 
##               w = 0.2111002
##               N = 216.2024
##              df = 2
##       sig.level = 0.05
##           power = 0.8
## 
## NOTE: N is the number of observations
\end{verbatim}

This indicates an effect size of w = 0.211 and sample size of 216.

Scenario 2 examines what would happen in the rate of exploratory
research in preregistered studies was the same as for standard reports:

\begin{verbatim}
##       SR   SO   RR   PR
## No  0.11 0.11 0.06 0.11
## Yes 0.14 0.14 0.19 0.14
\end{verbatim}

\begin{verbatim}
## 
##      Chi squared power calculation 
## 
##               w = 0.1775552
##               N = 305.6123
##              df = 2
##       sig.level = 0.05
##           power = 0.8
## 
## NOTE: N is the number of observations
\end{verbatim}

This indicates an effect size of w = 0.178 and sample size of 306.

Scenario 3 examines if the rate of exploratory research is higher in the
matched sample of traditional articles compared with the 2010
non-registered articles, while keeping the rate the same for
preregistered studies and Registered Reports.

\begin{verbatim}
##       SR   SO   RR   PR
## No  0.11 0.14 0.06 0.06
## Yes 0.14 0.11 0.19 0.19
\end{verbatim}

\begin{verbatim}
## 
##      Chi squared power calculation 
## 
##               w = 0.2832368
##               N = 120.0987
##              df = 2
##       sig.level = 0.05
##           power = 0.8
## 
## NOTE: N is the number of observations
\end{verbatim}

This indicates an effect size of w = 0.283 and sample size of 120.

Finally, Scenario 4 combines the two previous changes into a single
model:

\begin{verbatim}
##       SR   SO   RR   PR
## No  0.11 0.14 0.06 0.11
## Yes 0.14 0.11 0.19 0.14
\end{verbatim}

\begin{verbatim}
## 
##      Chi squared power calculation 
## 
##               w = 0.2327814
##               N = 177.8038
##              df = 2
##       sig.level = 0.05
##           power = 0.8
## 
## NOTE: N is the number of observations
\end{verbatim}

This indicates an effect size of w = 0.233 and sample size of 178.

Taken together, this set of analyses suggests a range of effect sizes
of, w = 0.178-0.283, and a range of sample sizes of, \emph{N} = 120-306.
We thus opted for a total sample size of 300 articles.

The following summary text will be inserted into the manuscript:

Full details of the power analysis are available at
\url{https://osf.io/4vbtq/}; here, we provide a summary. The arguments
made in the literature that registration will restrict exploratory
research has often used extreme language such as that it will ``put
researchers in chains'' (Scott, 2013), ``stifle discovery''
(Goldin-Meadow, 2016) and serves as a ``stranglehold'' on research
(McDermott, 2022). The severity of this language clearly implies a large
effect, that we would see a dramatically lower rate of exploratory work
in registered compared to non-registered studies. The only available
evidence, however, indicates a moderate difference in the other
direction, with a greater prevalence of exploratory research in
Registered Reports compared with standard reports (O'Mahony, 2023). The
effect size in O'Mahony (2023) is not reported, but we can use the
reported test statistic, ꭕ2(1, N = 510) = 16.19, p \textless{} 0.001, to
calculate it as ɸ = 0.18. This corresponded to an approximate difference
in prevalence of 20\%.

It is difficult to say what effect size is the smallest that would be
meaningful. Our goal in this project, however, is to address the extreme
claims about the damaging impact of registration. Thus, absent any
evidence, we would thus primarily be interested in large effects.
O'Mahony (2023) provides evidence of a moderate effect, albeit in the
other direction. That this effect was found in a registered thesis using
a large sample of articles suggests that there is low risk of bias.
Thus, we used ɸ = 0.18 as our estimate for the present study. The
pwr.chisq.test function from the pwr package in R (Champely et al.,
2020) indicated a necessary total sample size of 304 for a 2x2
contingency table (registered vs.~not registered), with alpha = .05 and
power = .80. That assumes that we are collapsing the two registered
articles types together and two traditional articles types together and
comparing them against each other. If, instead, we compared all four
articles types, but assumed that the pattern of exploratory research
would be the same across the two broad categories, then that would yield
an effect size of Cramer's V = 0.21 and a suggested sample size of 216.
Relaxing these assumptions in different ways leads to a range of effect
sizes from Cramer's V = 0.18-0.28 and a range of sample sizes from
\emph{N} = 120-306. Taken together, the power analysis indicates a total
sample of 300 articles will be sufficient for the aims of the study.

\end{document}
